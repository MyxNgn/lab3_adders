% Digital Logic Report Template
% Created: 2020-01-10, John Miller

%==========================================================
%=========== Document Setup  ==============================

% Formatting defined by class file
\documentclass[11pt]{article}

% ---- Document formatting ----
\usepackage[margin=1in]{geometry}	% Narrower margins
\usepackage{booktabs}				% Nice formatting of tables
\usepackage{graphicx}				% Ability to include graphics

%\setlength\parindent{0pt}	% Do not indent first line of paragraphs 
\usepackage[parfill]{parskip}		% Line space b/w paragraphs
%	parfill option prevents last line of pgrph from being fully justified

% Parskip package adds too much space around titles, fix with this
\RequirePackage{titlesec}
\titlespacing\section{0pt}{8pt plus 4pt minus 2pt}{3pt plus 2pt minus 2pt}
\titlespacing\subsection{0pt}{4pt plus 4pt minus 2pt}{-2pt plus 2pt minus 2pt}
\titlespacing\subsubsection{0pt}{2pt plus 4pt minus 2pt}{-6pt plus 2pt minus 2pt}

% ---- Hyperlinks ----
\usepackage[colorlinks=true,urlcolor=blue]{hyperref}	% For URL's. Automatically links internal references.

% ---- Code listings ----
\usepackage{listings} 					% Nice code layout and inclusion
\usepackage[usenames,dvipsnames]{xcolor}	% Colors (needs to be defined before using colors)

% Define custom colors for listings
\definecolor{listinggray}{gray}{0.98}		% Listings background color
\definecolor{rulegray}{gray}{0.7}			% Listings rule/frame color

% Style for Verilog
\lstdefinestyle{Verilog}{
	language=Verilog,					% Verilog
	backgroundcolor=\color{listinggray},	% light gray background
	rulecolor=\color{blue}, 			% blue frame lines
	frame=tb,							% lines above & below
	linewidth=\columnwidth, 			% set line width
	basicstyle=\small\ttfamily,	% basic font style that is used for the code	
	breaklines=true, 					% allow breaking across columns/pages
	tabsize=3,							% set tab size
	commentstyle=\color{gray},	% comments in italic 
	stringstyle=\upshape,				% strings are printed in normal font
	showspaces=false,					% don't underscore spaces
}

% How to use: \Verilog[listing_options]{file}
\newcommand{\Verilog}[2][]{%
	\lstinputlisting[style=Verilog,#1]{#2}
}




%======================================================
%=========== Body  ====================================
\begin{document}

\title{ELC 2137 Lab 03: Adders}
\author{My Nguyen}

\maketitle

\section*{Summary}

This lab's purpose is to create a circuit that implement a half, full, and 2-bit adder. To implement a half adder, a XOR gate is used to store the sum between two inputs and an AND gate is used to store the carry between two inputs. To implement a full adder, create a first stage half adder then pass its sum and the carry input into a second stage half adder to find the $S$. Then pass the second stage half adder carry and the first stage half adder into a XOR gate to find the carry output. To implement a 2-bit adder, use a full bit adder to find the $S_1$, then use the carry output from the first full adder to as the carry input for a second full adder to find the $S_2$ and the carry output of the 2-bit adder.


\section*{Q\&A}
1.
\begin{center}
	\includegraphics[width=\textwidth]{"sheet"}
\end{center}

2.
\begin{figure}
	\centering
	\includegraphics[width=\textwidth,angle=90]{"full_adder"}
	\caption{Full Adder}
\end{figure}

3.
\begin{center}
	\begin{tabular}{c|c|c||c|c}
		$C_{in}$ & A & B & $C_{1}$ & $C_{2}$ \\
		\midrule
		0 & 0 & 0 & 0 & 0 \\
		0 & 0 & 1 & 0 & 0 \\
		0 & 1 & 0 & 0 & 0 \\
		0 & 1 & 1 & 1 & 0 \\
		\midrule
		1 & 0 & 0 & 0 & 0 \\
		1 & 0 & 1 & 0 & 1 \\
		1 & 1 & 0 & 0 & 1 \\
		1 & 1 & 1 & 1 & 0 \\
		\bottomrule
	\end{tabular}
	
\end{center}
From the truth table, we have prove by exhaustion that carry outputs of the first half adder and the second half adder cannot be high at the same time

4. We should use an AND gate for carry bits, because when adding two bits, a carry is only needed when both input is a "1" so an AND gate would be appropriate to indicate the carry bits.


\end{document}
